\begin{abstract}
	Aiming towards sustainable heat supply for residential/commercial buildings implies the necessity of decarbonizing heat production portfolios. Most decarbonization studies examine net-zero scenarios on a highly aggregated level using integrated assessment models (IAMs) with global coverage. To translate these high-level transformation pathways to policy measures at a local resolution, it is necessary to downscale results from an aggregated level to a higher granularity. This work’s core objective is to examine the local network topology of sustainable heat supply and to identify the trade-offs for heat supply companies between low-carbon energy carriers, a significant heat demand reduction by building renovation, and a heat network expansion integrating renewable technologies such as geothermal and green gas high-efficiently. A two-stage analysis is proposed, including a downscaling algorithm for using IAM results for obtaining high spatial granularity using a novel downscaling technique accounting for the infrastructure requirements of centralized heat supply options and population density as criteria, and a benchmarking assessing network-based heat supply topologies. Using Austria as a case study, we downscale values projected by different decarbonization storylines from the H2020 openENTRANCE project. Results indicate that sustainable heat networks achieve only lower heat densities compared to existing networks, thus reducing infrastructure to supply ratio efficiency.
	% system analysis-based policy recommendations 
\end{abstract}

\begin{keyword}
\end{keyword}
\end{frontmatter}