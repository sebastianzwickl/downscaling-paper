\section{Conclusions and recommendations}\label{conclusions}
\added{The} sustainable energy transition requires methods to bridge the gap between global decarbonization pathways and the \replaced{corresponding}{resulting necessary} measures at local levels. This work emphasizes the development of different downscaling \replaced{techniques}{algorithms}, which we apply to the European heating sector under several \replaced{scenarios}{storylines} in line with the Paris Agreement \added{and its remaining CO\textsubscript{2} budget}. \added{We use the cost-effective European heat supply from the aggregate model GENeSYS-MOD to analyze results at the community level in Austria. The remaining European CO\textsubscript{2} budget (and related CO\textsubscript{2} prices) in line with the 1.5°C climate target is considered by the GENeSYS-MOD results. The downscaling includes the technology-specific infrastructure requirements for the highly efficient usage of heat sources in district heating}.\vspace{0.3cm}

\added{We found that the cost-effective heat supply at the European and national level in 2050 implies that district heating covers parts of the heat demand in four of the thirty-five sub-regions in Austria. Furthermore, the results demonstrate that district heating continues to be picking cherries from beneficial areas (i.e., densely populated with high heat densities) as only some communities of the four mentioned areas are supplied by district heating. Nevertheless, not all district heating networks and supply areas in 2050 reach the heat density required for economic and technical efficiency from today’s techno-economic perspective and industry benchmarks. This heat density gap (mainly driven by a significant reduction of heat demands by building renovation measures) poses a challenge for district heating but can be reduced by the optimal allocation of large-scale heat pump (air) generation into district heating.}\vspace{0.3cm}

\added{We anticipate our work as a starting point for discussing the role of district heating enabling large-scale, highly efficient, and local integration of renewable heat sources such as geothermal, synthetic gas, hydrogen, and waste in sustainable energy systems with decreasing heat demands. Further research should follow on how obtained district heating networks and their heat densities (incl. the generation of large-scale heat pump (air) units) could be returned into more aggregate models, such as GENeSYS-MOD, in the sense of a feedback loop. That allows refining assumptions in the large-scale models, which in turn will increase the plausibility and realism of pathways at the European level.}

