\section{Conclusions and recommendations}\label{conclusions}
The sustainable energy transition requires methods to bridge the gap between global decarbonization pathways and the corresponding measures at local levels. This work emphasizes the development of a downscaling technique \added{based on a simplified optimization model}, which we apply to the European heating sector under several scenarios in line with the Paris Agreement and its remaining CO\textsubscript{2} budget. Next, we use the cost-effective European heat supply from the aggregate model GENeSYS-MOD to analyze results at the local administrative unit level in Austria. \added{The implemented method allows to disaggregate heat supply from the country to the community level, which 
contributes to the linkage between the modelers and the practitioners perspective.} The remaining European CO\textsubscript{2} budget (and related CO\textsubscript{2} prices) in line with the 1.5°C climate target is considered by the GENeSYS-MOD results. The downscaling includes \added{the claim that the renewable heat sources geothermal, green gases (synthetic gas and hydrogen), waste, and large-scale heat pumps (air-sourced) are used in district heating.}\vspace{0.3cm}

We found that the cost-effective heat supply at the European and national level in 2050 implies that district heating covers \added{heat demand in 68 communities in Austria in 2050, which corresponds to 6\% of the total number of communities.} The results demonstrate that district heating continues to be picking cherries from beneficial areas (i.e., densely populated with high heat densities). However, the reduction of heat densities compared to today's values is mainly driven by a significant reduction of heat demands by building renovation measures and poses a challenge for district heating in the future. \added{Nevertheless, the localization of district heating networks in the surrounding of urban areas indicates economic viability, too.}\vspace{0.3cm}

\added{In view of comparing different scenarios in this work, the results indicate that the aggregate model GENeSYS-MOD is capable of handling planning approaches for decarbonization of energy systems. Particularly, the policy push in one of the scenarios (Directed Transition) is also reflected in the determined local heat densities of district heating. This can be seen in particular by the projected share of both district heating and large-scale heat pumps in GENeSYS-MOD's results.}\vspace{0.3cm}

We anticipate our work as a starting point for discussing the role of district heating as an enabler for large-scale, highly efficient, and local integration of renewable heat sources such as geothermal, synthetic gas, hydrogen, and waste in sustainable energy systems with decreasing heat demands. Further research should follow on how obtained district heating networks and their heat densities (incl. the generation of large-scale heat pump (air) units) could be returned into more aggregate models, such as GENeSYS-MOD, in the sense of a feedback loop. That allows refining assumptions in the upper-level large-scale models, which in turn will increase the plausibility and realism of pathways at the European level.


