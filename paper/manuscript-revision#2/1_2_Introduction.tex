\section{Introduction}
To implement the pathway in line with the Paris Climate Agreement \cite{agreement2015paris} as analyzed by the IPCC's \emph{Special Report on Global Warming of 1.5°C} (SR15) \cite{book}, the European Commission set deep decarbonization targets together with national governments. In particular, the \textit{EU Green Deal} describes the concrete goals in Europe, namely, a climate-neutral and resource-conserving economy and society \cite{kemfert2019green}. The overarching goal is to reduce CO\textsubscript{2} emissions to net-zero and hence achieve climate neutrality by 2050. The principles of a net-zero, decarbonized society are based on three key points: (i) reduction of the energy demand \cite{grubler2018low}, (ii) deployment and generation of renewable energy technologies \cite{bakhtavar2020assessment}, and (iii) an increase in efficiency regarding the provision of energy services and the associated optimal utilization of sustainable energy sources.\vspace{0.5cm}

To achieve these long-term ambitions, the European Commission recently presented \textit{Fit for 55}, a roadmap with specific actions and targets until 2030. This program commits to a \SI{55}{\%} reduction in CO\textsubscript{2} emissions in 2030 compared to those in 1990 \cite{european_commission_european_2019}. The concrete measures affect almost all sectors of the energy system and should lead to a significant efficiency improvement and a massive overall reduction in fossil fuels. It implies, among others, binding annual targets to reduce energy consumption and to extend the already established EU emissions trading system (EU ETS) to new sectors. In addition to transportation, in the future, the building sector will be part of the EU ETS. In the building sector, using the annual anchored emissions reduction means a defined roadmap to complete decarbonization of the heating and cooling demand. In this paper, we look at what deep decarbonization of building heat demands may look like in 2050 in Austria and the implications of the corresponding sustainable energy mix for district heating.

\subsection{Implications of decarbonization on the heating sector}
The scope of changes required by 2030/2050 in the heating sector becomes even clearer at the national level. In Europe, the share of renewable energies in the heating and cooling sector in 2018 is only just above \SI{20}{\%} on average \cite{eurostat_reference}. In Austria, it reaches 34\%. However, fossil fuels continue to dominate there as well. In 2015, the heat demand for low-temperature heat services in Austria was approximately \SI{96}{TWh}. \added{This heat volume encompasses low-temperature heat demand of residential buildings (domestic spacial heating demand, calculated on the basis of the outdoor temperature), industrial heat demand below 100°C (e.g., food sector, machinery, and wood), and process heat demand \cite{burandt2018genesys}.} In the residential building sector, natural gas, oil, and coal account for almost 45\% of space heating and hot water demand  \cite{oesterreichsenergie}. The share of district heating reaches almost 15\%, and more than one million households are connected to district heating networks. According to \cite{statisitik2016}, the total heat production from district heating was around \SI{24}{TWh} \added{in 2016. Thereby, the share of renewable energy was 45\%. Besides, the share of waste sources was 9\%. In 2018, district heating supplied 18\% of the total heat demand in the residential building and service sector with a share of 48\% renewable heat sources. Thereby, the amount of district heating was} \SI{20}{TWh}. \added{Table \ref{tab:numbers} provides an rough overview of the Austrian heat market, as it shows the proportion per heat source/generation technology on the total heat demand for space heating and hot water.}

\begin{table}[h]
	\centering
	\resizebox{1\textwidth}{!}{% use resizebox with textwidth
		\renewcommand{\arraystretch}{1.1}
		\begin{tabular}{lcc}
			\toprule 
			& Proportion in \% & Abs. number\\\cmidrule(rl){2-2}\cmidrule(rl){3-3}
			Heat source/technology& on space and hot water demand & of households supplied\\\hline
			Biomass & 28.3 & 725,439\\
			Natural gas & 26.5 & 913,448\\
			Oil & 17.2 & 626,109\\
			District heating & 14.6 & 1,112,734\\
			Direct electric & 8.2 & 210,648\\
			Heat pumps & 3.0 & \multirow{2}{*}{294,075}\\
			Solarthermal & 1.9 & \\
			Coal & 0.4 & 7,640\\
			\bottomrule
	\end{tabular}}
	\caption{\added{Proportion of heat sources/generation technologies on the total heat demand (space and hot water) and absolute number of households supplied for Austria in 2017. The number of households supplied by heat pumps and solarthermal is given in total. Source: \cite{oesterreichsenergie}.}}
	\label{tab:numbers}
\end{table}

Nevertheless, of the nearly 4,000,000 residential dwellings in Austria, more than one million are heated with natural gas, and more than 500,000 are heated with oil \cite{statistik_austria}. If these heating systems are converted to renewable energy supply by 20\replaced{4}{5}0, this corresponds to a retrofitting of \added{more than} \replaced{80}{50},000 units per year, or more than \replaced{225}{130} per day - only in Austria. To achieve this goal, measures that go beyond the electrification of heat supply are necessary, which may require an expansion of district heating networks. This holds true even when substantial heat saving measures are implemented \cite{jalil2018spatially}.\vspace{0.3cm}

In Europe, good conditions for district heating exist \cite{persson2019heat}, especially in the provision of heat services in densely populated or urban areas \cite{inage2020development} because of high heat densities that are found there. In addition to heat density, the connection rate is a key factor determining the efficiency of district heating/cooling networks and thus their implementation. In Austria, a benchmark of \SI{10}{GWh \per km^2} at a connection rate of \SI{90}{\%} is currently used when deciding whether to supply an area with district heating\footnote{\url{http://www.austrian-heatmap.gv.at/ergebnisse/}}. This reference value \added{considers the area effectively supplied by district heating and not the total area. Thus, the exclusion of land areas that contain woodland, mountain, agricultural and other low heat density areas is crucial. The reference/benchmark value} is in line with findings regarding district heating networks also from the Scandinavian region (Denmark, Sweden, and Finland) \cite{zinko2008district}. These are rough estimates, but they do allow an initial assessment of the economic viability or feasibility of a district heating network. In a detailed consideration and evaluation of district heating networks, numerous factors play a decisive role. For example, the design and topology of district heating networks demonstrates significant impact on their cost-effectiveness \cite{nussbaumer2016influence, zvoleff2009impact}. In addition, the cost-optimized heat supply is also influenced by the location of heat generation units/sources within the networks \cite{laasasenaho2019gis}. \added{The influence of the connection rate and linearly decreasing heat densities on the profitability of district heating networks is investigated in \cite{nilsson2008sparse} and \cite{dochev2018analysing}. The study in \cite{bordin2016optimization} presents an optimization approach for district heating strategic network design. Further works also evaluate the impact of the heating system topology on energy savings \cite{allen2020evaluation}.} When examining the economic viability of district heating networks, building renovation measures must also be taken into account \cite{andric2018impact}. Recently, the results in \cite{hietaharju2021stochastic} show that a $2-3$\SI{}{\%} building renovation rate per year results in a $19-28$\SI{}{\%} decrease of the long-term district heating demand, which consequently also reduces the heat densities of district heating networks. However, studies show that a reduction in heat density is not necessarily a barrier to district heating networks \cite{persson2011heat}. For example, energy taxes which can certainly be expected in the future (e.g., higher taxes on fossil fuels) can improve the profitability of sparse district heating networks \cite{reidhav2008profitability}. Following these considerations and in light of ambitious CO\textsubscript{2} reduction targets assumptions exists that rising CO\textsubscript{2} prices exhibit a similar effect. However, this is valid only in the case of deep decarbonization of the generation mix feeding into district heating networks. In general, a variety of alternatives to decarbonize the energy mix of district heating networks exists. Among others, geothermal \cite{kyriakis2016towards}, biomass \cite{di2014low}, waste \cite{hiltunen2020highly} and heat recovery from industrial excess heat \cite{buhler2017industrial} are likely to be the primary heat sources in sustainable district heating networks. Eventually, the increasing cooling demand and the co-design of district heating and cooling networks can also increase the economic viability of these and counteract the reduction of heat density from an economic point of view \cite{zhang2021economic}.

\subsection{Implications of large-scale numerical model results at the local level}
For quantifying solutions of complex planning problems, researchers use numerical models. In general, these models strike a balance between complexity and aggregation. Integrated assessment models (IAMs) are large numerical models covering complex interrelationships between climate, society, economics, policy, and technology \cite{dowlatabadi1995integrated}. Particularly, IAM contribute to the understanding of global energy decarbonization pathways \cite{wilkerson2015comparison}. Evaluating and discussing IAM involves, among others, the appropriate level of regional (spatial) aggregation of countries in the modeling analysis \cite{schwanitz2013evaluating}. Generalizing this aspect reveals an aspect already known but essential in the context of large numerical models. Setting priorities regarding the level of detail becomes necessary for modelers, which inevitably creates trade-offs in the analysis regarding the granularity of temporal, spatial, and other dimensions \cite{gargiulo2013long}. Accordingly, IAMs should increasingly be supplemented with other models and analytical approaches \cite{gambhir2019review}. Not least for this reason, large-scale detailed energy systems models also play a significant role in the analysis of energy systems in the context of climate change. Compared to IAMs, they more strongly emphasize the level of detail in terms of techno-economic characteristics. However, the lack of granularity remains; these global systems models consider only a highly aggregated spatial resolution. To name just two selected approaches, PRIMES \cite{capros2012model} and GENeSYS-MOD \cite{loffler2017designing} are aggregate energy system models focusing on the European energy system with a spatial resolution at the country level. Further approaches are needed to disaggregate results obtained at the country level to finer scales, such as districts, neighborhoods, and other local levels. In this context, a novel approach in the context of merging local activities/behavior in sustainable local communities into a large energy system model (bottom-up linkage) is presented in \cite{backe2021heat}. In this study, local flexibility options are integrated into the global energy system model EMPIRE, which provides, in principle, only country level resolution. This and other work confirms the emerging trend of making top-down and bottom-up linkages between different spatial-temporal levels of resolution to drive decarbonization across all sectors.\vspace{0.3cm}

\subsection{Objective and contribution of this work}
Against this background, the core objective of this work is downscaling European decarbonization scenarios of the heating sector to the community levels serving end-users in 2050. In particular, downscaling considers the highly efficient and local use of sustainable heat sources in district heating (e.g., geothermal sources, co-firing synthetic gas and hydrogen in cogeneration plants, and large-scale waste utilization). In addition, the topography of district heating networks is of particular importance and plays a crucial role in applied downscaling. This allows estimates of realistic  decarbonized district heating networks in 2050 to be obtained, which can be compared with existing networks. Thereby, the heat density of district heating networks serves as a comparative indicator and permits a rough estimation of the changes needed for district heating networks considering the 1.5°C and 2.0°C climate target. An Austrian case study is conducted, downscaling the cost-effective results of the heating sector in 2050 from the large numerical energy system model GENeSYS-MOD, from the country to the community\added{/local administrative unit} levels. \added{In general, GENeSYS-MOD exhibits a focus on generic heat supply options based on primary energy sources, rather than specific local excess heat sources that is in general the fundamental idea of district heating. Accordingly, this study can be seen as an attempt for a stress test applying GENeSYS-MOD's heat supply in the context of district heating.} The GENeSYS-MOD results, and thus the values to be downscaled implicitly, include the remaining European CO\textsubscript{2} budget in line with the 1.5°C and 2.0°C climate target.\vspace{0.3cm}

The method applied (section \ref{methodology}) consists of \added{a simplified optimization model, which computes the amount of district heating at the local (community) level.} Section \ref{results} presents and discusses the results of this work. \added{First, the heat supply in a representative region is presented in section \ref{results1} and \ref{results2}. The projected heat densities at the community level in 2050 are presented in section \ref{results3} and \ref{results4}.} Finally, section \ref{conclusions} concludes this work and provides an outlook for future work.