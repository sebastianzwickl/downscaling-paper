\section{Introduction}
Climate change and its already apparent negative effects are likely to be the greatest challenge of humanity - both today and in the next decades \cite{karl2003modern}. What is encouraging in this regard, however, is that humanity has recently built a deep understanding of existing natural \cite{griscom2017natural} and human-influence solutions to mitigate climate change and global warming. At the same time, studies show that two-thirds of impacts with regard to the increase of global temperature (atmospheric and ocean) can be attributed to anthropogenic activities \cite{hansen2016assessing}. Thus, to speak of natural variability is far from justified. There is already a broad consensus to make it our prioritized concern to reduce anthropogenic emissions and get them down to net-zero rapidly.\newline

In the field of a sustainable energy system, the scientific community has produced a large number of strategies, recommendations, and studies aimed at contributing to the achievement of climate change mitigation and the clean energy transition. First and foremost, the reports of the Intergovernmental Panel on Climate Change (IPCC) \cite{edenhofer2011ipcc}, and the Paris Agreement \cite{agreement2015paris} but also many others. However, the concrete implementation to achieve the predefined goals still is lacking in many cases. For this reason, numerous studies go beyond and show roadmaps for the rapid decarbonization of the system. For example, Rockstr{\"o}m et al. \cite{rockstrom2017roadmap} conduct such a study and propose pathways for halving gross anthropogenic CO\textsubscript{2} emissions every decade. Other works go into more depth regarding optimal solutions for the decarbonization of individual energy services. There are relevant differences between the individual sectors of the energy system related to decarbonization. How a sustainable energy service can be provided in the different sectors must therefore be examined in detail. This perspective is supported by a large number of detailed decarbonization studies covering specific energy service needs (e.g., for the building sector Leibowicz et al. \cite{leibowicz2018optimal}, transport sector Pan et al. \cite{pan2018decarbonization}, and industries Habert et al. \cite{habert2020environmental}).\newline

Despite all the details associated with the sector-specific decarbonization strategies, the principles of a net-zero society base on three key points: (i) deployment and generation of renewable energy technologies (see, e.g., Bakhtavar et al. \cite{bakhtavar2020assessment} focusing on net-zero districts by deployment of renewable energy generation), (ii) reduction of the energy demand (see, e.g., Oshiro et al. \cite{oshiro2021enabling} analyzing the impact of energy service demand reduction on the decarbonization and Grubler et al. \cite{grubler2018low} investigating a low energy demand decarbonization scenario), and (iii) increase in efficiency regarding the provision of energy services and the associated optimal utilization of sustainable energy sources. 

The third point (iii) includes, among others, two main aspects, namely, on the one hand, that potentials of renewable resources are exploited locally and on the other hand that energy carriers with various fields of application are utilized with the highest possible efficiency. We like to refer to just a few selected references without claiming to be exhaustive and focus here on hydrogen as one example of an energy carrier with high potentials in sustainable energy systems and a significant bandwidth of efficiency in terms of its generation and use. Van Ruijven et al. \cite{van2007potential} highlight that the introduction of hydrogen in global energy systems only leads to lower emissions with high end-use efficiency and low-carbon production. Van Ressen \cite{van2020hydrogen} systematically investigates the possibilities and challenges of hydrogen and discusses extensively its role in the energy transition.  Thus, it is necessary to develop optimal strategies ensuring the utilization of renewable energy sources prioritized and in the most efficient way related to the provision of energy service needs.\newline 

In many cases when it comes to the question of optimal solutions, academic uses numerical models. In general, these models strike a balance between complexity and aggregation. Integrated assessment models (IAMs) are large numerical models covering complex interrelations between climate, society, economics, policy, and technology. Dowlatabadi \cite{dowlatabadi1995integrated} provided 1995 a fundamental review on IAMs focusing on their role in the context of climate change. Krey et al. \cite{krey2019looking} discuss and systematically compare different IAMs. Harmsen et al. \cite{harmsen2021integrated} elaborates on the modeling behaviour of IAMs. Wilkerson et al. \cite{wilkerson2015comparison} and van Vuuren et al. \cite{van2016carbon} deal with IAMs and their role in understanding global energy decarbonization pathways. In particular, both studies examine CO\textsubscript{2} budget and price developments. Huppmann et al. \cite{huppmann2019messageix} provide an open framework for integrated and cross-cutting analysis of energy, climate, the environment, and sustainable development. Schwanitz \cite{schwanitz2013evaluating} evaluates IAMs of global climate change and discusses, among others, the appropriate level of regional (spatial) aggregation of countries in the modeling analysis. Generalizing this aspect reveals an aspect already known but essential in the context of large numerical models. It becomes necessary for modelers to set priorities regarding the level of detail, which inevitably creates trade-offs in the analysis regarding the granularity of the temporal, spatial, and other dimensions. Gambhir et al. \cite{gambhir2019review} also highlight this aspect of aggregation bias in their critical review of IAMs. They propose, among others, that IAMs should be increasingly be supplemented with other models and analytical approaches. Not least for this reason, (large) energy models also play a significant role in the analysis of energy systems in the context of climate change. Compared to IAMs, they more strongly emphasize the level of detail in terms of techno-economic characteristics (see the review of modeling tools of energy systems in \cite{ringkjob2018review}). However, the lack of granularity remains resulting, that these (global) energy models consider only a highly aggregated spatial resolution. To name just two selected approaches, Capros et al. \cite{capros2012model} (PRIMES) and Löffler et al. \cite{loffler2017designing} (GENeSYS-MOD) provide energy system models focusing on the European energy system with a spatial resolution on the country level. Further approaches are needed to disaggregate results obtained at the country level to finer scales, such as districts, neighborhoods, and other local levels. In this context, Backe et al. \cite{backe2021heat} provided a novel approach in the context of merging local activities/behavior in sustainable local communities into a large energy system model (bottom-up linkage). In their study, they integrated local flexibility options into the global energy system model EMPIRE, which provides in principle only country-level resolution. This and other work confirms the emerging trend of making top-down and bottom-up linkages between different spatial-temporal levels of resolution to drive decarbonization across all sectors.\newline

Against this background, the core objective of this work is the downscaling of decarbonization scenarios of the residential and commercial heating sector, taking into account the infrastructure requirements of heat generation technologies/sources from the country to the local level. In particular, the prioritized preference of heat sources in centralized heat networks plays a key role, ensuring highly efficient usage of heat sources covering heat service needs. The assessment of centralized heat networks using heat density as a criterion is important in this analysis.\newline

The method applied is the development of two scenario-independent downscaling techniques. As the first, an sequential downscaling approach is presented, dissaggregating from the country level to different sub-regions. Thereby, population density and the infrastructure requirements of heat technologies serve as additional criterion in the downscaling. As the second, an iterative downscaling algorithm is presented. This algorithm bases on graph-theory benchmarking and projects centralized heat supply on the local level. An Austrian case study is proposed, downscaling four different decarbonization scenarios, obtained from the large numerical energy system model GENeSYS-MOD, from the country to 2095 local districts. In particular, the focus is put on the year 2050.\newline 

This paper is organized as follows. Section \ref{methodology} explains the method including the description of the different downscaling techniques (Section \ref{pop} default downscaling method using population as criteria, Section \ref{alg1} sequential, and Section \ref{alg2} iterative downscaling). Section \ref{results} presents and discusses the results of this work. Section \ref{res:1} and \ref{res:2} showing heat generation by source on different spatial levels. Section \ref{res:3} and \ref{res:4} presenting centralized heat networks on a high spatial granularity. Section \ref{res:5} synthesizes the results of centralized heat networks and compares heat densities of centralized heat networks in 2050 with today's values. Section \ref{conclusions} concludes this work and provides an outlook for future work. 