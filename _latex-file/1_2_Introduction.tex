\section{Introduction}
Climate change and its already apparent negative effects are likely to be the greatest challenge of humanity - both today and in the next decades \cite{karl2003modern}. What is encouraging in this regard, however, is that we have recently built a deep understanding of existing natural \cite{griscom2017natural} and human-influence solutions to mitigate climate change and global warming. At the same time, studies show that two-thirds of impacts with regard to the increase of global temperature (atmospheric and ocean) can be attributed to anthropogenic activities \cite{hansen2016assessing}. Thus, to speak of natural variability is far from justified. It is for that reason that there is already the broad consens to make it our prioritized concern to reduce anthropogenic emissions and get them down to zero.\newline

In the field of a sustainable energy system, the scientific community has produced a large number of strategies, recommendations and studies aimed at contributing to the achievement of the clean energy transition and thus a sustainable provision of energy services \cite{rockstrom2017roadmap}. Of course, there are differences between the individiual sectors of the energy system, which has led to the development of tailor-made strategies and recommendations to supply the different kinds of energy services. Leibowicz et al. \cite{leibowicz2018optimal} present optimal decarbonization pathways for the supply of energy services for urban residential buildings. Pan et al. \cite{pan2018decarbonization} focus on the decarbonization of the transport sector in China in the light of the Paris Agreement goals. Habert et al. \cite{habert2020environmental} provide an comprehensive analysis of the environmental impacts and decarbonization strategies in concrete industries. In addition, we recommend with regard to challenges in the decarbonization of the energy sector the recently published work by Papadis and Tsataronis \cite{papadis2020challenges}. Despite all the details associated with the sector-specific decarbonization strategies, the principles of a net-zero society base on three key points: (i) a massive deployment/generation of renewable energy technologies (e.g., Bakhtavar et al. \cite{bakhtavar2020assessment} focusing on net-zero districts), (ii) a significant reduction of the energy demand (e.g., Oshiro et al. \cite{oshiro2021enabling} conducting an analysis on the impact of energy service demand reduction on the decarbonization), and (iii) an increase in efficiency in the context of the provision of energy services and the associated optimal utilization of energy carriers and sources. The latter point includes, among others, two detailed aspects, namely, on the one hand, that potentials of renewable resources are exploited locally and on the other hand that energy carriers with various fields of application are utilized with the highest possible efficiency. We would like to refer to just a few selected references without claiming to be exhaustive and focus here on hydrogen as one example of an energy carrier with high potentials in sustainable energy systems and a significant bandwidth of efficiency in terms of its generation and use. Van Ressen \cite{van2020hydrogen} systematically investigates the possibilities and challenges of hydrogen and discusses extensively its role in the energy transition. Van Ruijven et al. \cite{van2007potential} highlight that the introduction of hydrogen in global energy systems only leads to lower emissions with high end-use efficiency and low-carbon production. Therefore, it is necessary to think also about where we use the limited amounts of renewable energy carriers (such as renewable-based hydrogen or biomass, etc.) in the most efficient (optimized) way.\newline

In many cases when it comes to the question of optimal solutions, academic uses numerical models. In general, these models strike a balance between complexity and aggregation. Integrated assessment models (IAMs) are large-numerical models covering complex interrelations between climate, society, economics, policy and technology. Dowlatabadi \cite{dowlatabadi1995integrated} provided 1995 a fundamental review on IAMs focusing on their role in the context of climate change. Krey et al. \cite{krey2019looking} discuss and systematically compare 15 different IAMs. Wilkerson et al. \cite{wilkerson2015comparison} are heading in a similar direction focusing in their comparison on the impact of carbon price. Van Vuuren et al. \cite{van2016carbon} focus in their analysis on IAMs in the context of carbon budgets and energy transition pathways. Huppmann et al. \cite{huppmann2019messageix} provide an open framework for integrated and cross-cutting analysis of energy, climate, the environment, and sustainable development. Schwanitz \cite{schwanitz2013evaluating} evaluates IAMs of global climate change and discusses, among others, the appropiate level of regional (spatial) aggregation of countries in the modeling analysis. Generalizing this aspect reveals an aspect already known but essential in the context of large-numerical models. It becomes necessary for modelers to set priorities regarding the level of detail, which inevitably creates trade-offs in the analysis regarding the granularity of the temporal, spatial, and other dimensions. Gambhir et al. \cite{gambhir2019review} also highlight this aspect of aggregation bias in their critical review of IAMs. They propose, among others, that IAMs should be increasingly be supplemented with other models and analytical approaches. Not least for this reason, (large) energy models also play a significant role in the analysis of energy systems in the context of climate change. Compared to IAMs, they again emphasize the level of detail in terms of techno-economic characteristics (see the review of modeling tools of energy systems in \cite{ringkjob2018review}). However, the lack of granularity remains as a result that these (global) energy models consider only a highly aggregated spatial resolution. To name just two selected approaches, Capros et al. \cite{capros2012model} (PRIMES) and Löffler et al. \cite{loffler2017designing} (GENeSYS-MOD) provide energy system models focusing on the European energy system with a spatial resolution on the country. Thus, further approaches are needed to link development/results/insights generated at the country level with those at the local level in this and many other cases. In this context, Backe et al. \cite{backe2021heat} provided a novel approach in the context of merging local activities/behaviour in sustainable local communities into a large energy system model (bottom-up linkage). They integrated local flexibility options into the global energy system model EMPIRE, which provides in principle only country level resolution.\newline

Against this background, the core objective of this work is to downscale global energy system model results from the country to a high spatial granularity. We focus on the heating sector 2050 and consider as one of the main novelties in this work the infrastructure needs and requirements of sustainable heat generation sources in the downscaling. In particular, we investigate the modeling results of GENeSYS-MOD in the context of four different deep decarbonization scenarios developed in the H2020 openENTRANCE. We use downscaling as a method to estimate centralized heat networks on a high spatial granularity in 2050 and use heat density as criteria comparing the downscaled centralized heat networks with existing ones today.\newline

The method applied is a novel downscaling technqiue, which includes two separate subsequent downscaling methods. As the first, we introduce an iterative downscaling techniques which is used to dissaggregate heat generation from the country to different sub-regions. Thereby, we use population density and the infrastructure requirements of heat technology sources as additional criterias. As the second, we introduce an sequential downscaling algorithm. Thereby, we use graph-theory benchmarking techniques to improve centralized heat networks on the local level. We use an Austrian case study to demonstrate our approach and focus on the low-temperature heat supply and corresponding centralized heat networks.\newline

This paper is organized as follows:























%3) was es daher bedarf ist methode um IAM/energy models auf lokale Ebene herunterzubrechen da sich implizit damit handlungsbedarf ergibt im kontext von 
%regulatory interventions
%shifted the economics
%governmental regulation
%establishment of local renewable generation
%starke implikationen auf lokaler ebene auf gesetzgebung, förderwesen, regulierung, jedes einzelnen