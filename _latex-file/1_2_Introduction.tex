\section{Introduction}
The Paris Climate Agreement sets the global framework for mitigating climate change \cite{agreement2015paris}. It stipulates that the increase in the global average temperature should be kept well below \SI{2}{\degreeCelsius} compared to 1990. In addition, further measures are developed, aiming at a maximum increase of \SI{1.5}{\degreeCelsius}. However, it is also about humanity adapting to the negative effects of climate change that are already being felt. The IPCC Special Report on \SI{1.5}{\degreeCelsius} (SR1.5) summarizes the state of scientific knowledge globally on the consequences of \SI{1.5}{\degreeCelsius} global warming \cite{edenhofer2011ipcc}. Besides climate change mitigation measures, global emission pathways required for this are described.\newline

To implement the Paris Climate Agreement and the SR1.5, the European Commission has set deep decarbonization targets together with national governments. In particular, the EU Green Deal describes the concrete goals in Europe, namely a climate-neutral and resource-conserving economy and society. The overarching goal is emissions neutrality in 2050. To achieve this long-term ambition, the European Commission recently presented "Fit for 55", a concrete roadmap to 2030. This program commits to a 55 percent reduction in emissions in 2030 compared to 1990. The concrete measures affect almost all sectors of the energy system and should lead to a significant efficiency improvement and a massive overall reduction in fossil fuels. It implies, among others, binding annual targets for reducing energy consumption and an extension of the already established EU emissions trading system (EU ETS) to new sectors. In addition to transportation, the building sector will also be part of the EU ETS in the future. A separate new emissions trading system for fuel supply in these sectors will be introduced. In the buildings sector, through the annual anchored emissions reduction, this means a set roadmap to complete decarbonization of heating and cooling demand, as the two reasons for emissions in this sector. In this paper, we look at what deep decarbonization of building heating demand may look like in 2050 and the implications of the sustainable energy mix for centralized heating networks.\newline

The scope of the changes required by 2030/2050 in the heating sector become even clearer at the national level. The average share of renewable energies in the heating and cooling sector is only just above \SI{20}{\%} on average for all EU member states\footnote{\url{https://ec.europa.eu/eurostat/web/products-eurostat-news/-/ddn-20200211-1}}. It is in fact higher in some countries, for example in Austria, where it is \SI{34}{\%}. However, fossil fuels continue to dominate there as well. Of the nearly 4 million residential dwellings in Austria, more than 900 thousand are heated with natural gas, and more than 500 thousand with oil. If these heating systems are changed to renewable energy by 2050, this corresponds to a retrofitting of 50 thousand appliances per year, or more than 130 per day - only in Austria.\newline 

However, the concrete implementation to achieve predefined climate change mitigation goals still is lacking in many cases. For this reason, numerous studies go beyond and show roadmaps for the rapid decarbonization of the system. For example, Rockstr{\"o}m et al. \cite{rockstrom2017roadmap} conduct such a study and propose pathways for halving gross anthropogenic CO\textsubscript{2} emissions every decade. Other works go into more depth regarding optimal solutions for the decarbonization of individual energy services. There are relevant differences between the individual sectors of the energy system related to decarbonization. How a sustainable energy service can be provided in the different sectors must therefore be examined in detail. This perspective is supported by a large number of detailed decarbonization studies covering specific energy service needs (e.g., for the building sector Leibowicz et al. \cite{leibowicz2018optimal}, transport sector Pan et al. \cite{pan2018decarbonization}, and industries Habert et al. \cite{habert2020environmental}).\newline

Despite all the details associated with the sector-specific decarbonization strategies, the principles of a net-zero society base on three key points: (i) deployment and generation of renewable energy technologies (see, e.g., Bakhtavar et al. \cite{bakhtavar2020assessment} focusing on net-zero districts by deployment of renewable energy generation), (ii) reduction of the energy demand (see, e.g., Oshiro et al. \cite{oshiro2021enabling} analyzing the impact of energy service demand reduction on the decarbonization and Grubler et al. \cite{grubler2018low} investigating a low energy demand decarbonization scenario), and (iii) increase in efficiency regarding the provision of energy services and the associated optimal utilization of sustainable energy sources. The third point (iii) includes, among others, two main aspects, namely, on the one hand, that potentials of renewable resources are exploited locally and on the other hand that energy carriers with various fields of application are utilized with the highest possible efficiency. We like to refer to just a few selected references without claiming to be exhaustive and focus here on hydrogen as one example of an energy carrier with high potentials in sustainable energy systems and a significant bandwidth of efficiency in terms of its generation and use. Van Ruijven et al. \cite{van2007potential} highlight that the introduction of hydrogen in global energy systems only leads to lower emissions with high end-use efficiency and low-carbon production. Van Ressen \cite{van2020hydrogen} systematically investigates the possibilities and challenges of hydrogen and discusses extensively its role in the energy transition. Recently, Böhm et al. \cite{bohm2021power} comprehensively elaborate on hydrogen-related synergies and its role in sustainable heat supply. Thus, it is necessary to develop optimal strategies ensuring the utilization of renewable energy sources prioritized and in the most efficient way related to the provision of energy service needs.\newline 

In many cases when it comes to the question of optimal solutions, researcher uses numerical models. In general, these models strike a balance between complexity and aggregation. Integrated assessment models (IAMs) are large numerical models covering complex interrelations between climate, society, economics, policy, and technology. Dowlatabadi \cite{dowlatabadi1995integrated} provided 1995 a fundamental review on IAMs focusing on their role in the context of climate change. Krey et al. \cite{krey2019looking} discuss and systematically compare different IAMs. Harmsen et al. \cite{harmsen2021integrated} elaborates on the modeling behaviour of IAMs. Wilkerson et al. \cite{wilkerson2015comparison} and van Vuuren et al. \cite{van2016carbon} deal with IAMs and their role in understanding global energy decarbonization pathways. In particular, both studies examine CO\textsubscript{2} budget and price developments. Schwanitz \cite{schwanitz2013evaluating} evaluates IAMs of global climate change and discusses, among others, the appropriate level of regional (spatial) aggregation of countries in the modeling analysis. Generalizing this aspect reveals an aspect already known but essential in the context of large numerical models. It becomes necessary for modelers to set priorities regarding the level of detail, which inevitably creates trade-offs in the analysis regarding the granularity of the temporal, spatial, and other dimensions. Gambhir et al. \cite{gambhir2019review} also highlight this aspect of aggregation bias in their critical review of IAMs. They propose, among others, that IAMs should be increasingly be supplemented with other models and analytical approaches. Not least for this reason, (large) energy models also play a significant role in the analysis of energy systems in the context of climate change. Compared to IAMs, they more strongly emphasize the level of detail in terms of techno-economic characteristics (see the review of modeling tools of energy systems in \cite{ringkjob2018review}). However, the lack of granularity remains, that these (global) energy models consider only a highly aggregated spatial resolution. To name just two selected approaches, Capros et al. \cite{capros2012model} (PRIMES) and Löffler et al. \cite{loffler2017designing} (GENeSYS-MOD) provide energy system models focusing on the European energy system with a spatial resolution on the country level. Further approaches are needed to disaggregate results obtained at the country level to finer scales, such as districts, neighborhoods, and other local levels. In this context, Backe et al. \cite{backe2021heat} provided a novel approach in the context of merging local activities/behavior in sustainable local communities into a large energy system model (bottom-up linkage). In their study, they integrated local flexibility options into the global energy system model EMPIRE, which provides in principle only country-level resolution. This and other work confirms the emerging trend of making top-down and bottom-up linkages between different spatial-temporal levels of resolution to drive decarbonization across all sectors.\newline

Against this background, the core objective of this work is the downscaling of decarbonization scenarios of the residential and commercial heating sector, taking into account the infrastructure requirements of heat generation technologies/sources from the country to the local level. In particular, the prioritized preference of heat sources in centralized heat networks plays a key role, ensuring highly efficient usage of heat sources covering heat service needs. The assessment of centralized heat networks using heat density as a criterion is important in this analysis. An Austrian case study is proposed, downscaling values of the heating sector in 2050, obtained from the large numerical energy system model GENeSYS-MOD, from the country to 2095 local communities.\newline 

The method applied consists of three different scenario-independent downscaling techniques. As the first, proportional downscaling using population as proxy is used as reference (Section \ref{pop}). As the second, an sequential downscaling approach is presented, dissaggregating from the country level to the sub-region level. Thereby, population density and the infrastructure requirements of heat technologies serve as additional criterion in the downscaling (Section \ref{alg1}). And as the third, an iterative downscaling algorithm is presented. This algorithm bases on graph-theory benchmarking and projects centralized heat supply on the local (community) level (Section \ref{alg2}). Section \ref{results} presents and discusses the results of this work. Section \ref{res:1} and \ref{res:2} shows heat generation by source on different spatial levels. Section \ref{res:3} and \ref{res:4} presents centralized heat networks on a high spatial granularity. Section \ref{res:5} synthesizes the results of centralized heat networks and compares heat densities of centralized heat networks in 2050 with today's values. Section \ref{conclusions} concludes this work and provides an outlook for future work. 
%Gaffin et al. \cite{gaffin2004downscaling} focus in their analysis on downscaling of socio-economic projections from the IPCC Special Report on emission scenarios. See also Gidden et al. \cite{gidden2019global} focusing on global emission pathways under different socio-economic scenarios. 
%Michel et al. \cite{michel2021climate} recently published a study where they focus on downscaling of climate change scenarios on the country level and proposed an enhanced temporal downscaling approach. Spatial downscaling to the country level is exemplarily shown by Sferra et al. in \cite{sferra2019towards}.