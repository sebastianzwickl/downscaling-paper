\section{Conclusions and recommendations}\label{conclusions}
Sustainable energy transition requires methods to bridge the gap between global decarbonization plans and the resulting necessary measures at the local level. This work emphasizes the development of different downscaling algorithms in general, and the downscaling of the Austrian heating sector (residential and commercial) under the 1.5°C climate target to the community and grid levels in particular, considering technology-specific infrastructure requirements for the highly efficient usage of heat sources.\vspace{0.3cm}

We found that the prioritized perspective of efficiency and local utilization of renewable heat sources leads to a crucial treatment of the further development of district heating networks in the decarbonized Austrian heat supply toward 2050. This implies small-scale ($<\SI{1}{TWh}$) and large-scale ($>\SI{12}{TWh}$) district heating networks in terms of the amount of heat delivered. The results demonstrate that particularly densely populated areas are still beneficial supply areas for district heating networks and offer adequate heat densities. Nevertheless, most district heating networks in 2050 (seven of eight) will not reach the heat density benchmarks of today's networks and have a significant heat density gap. However, considering the increasing importance of local renewable heat sources feeding into district heating networks, we assume that these centralized networks will become required in the future and crucial in the decarbonization of the heating sector.\vspace{0.3cm}

We anticipate our work as a starting point, discussing the role of centralized heat networks as an infrastructure hub in the light of enabling large-scale, highly efficient, and local integration of renewable heat sources (such as biomass/waste, hydrogen, ground-sourced heat pumps, or geothermal units). In particular, we see a need for further research on the trade-off analysis between the efficiency/local integration of heat sources and the cost-intensive deployment of district heating networks. Future work may elaborate on the increasing cooling demand and how the cooperative design of district heating and cooling networks can contribute to the profitability of centralized heating and cooling infrastructure. 