\section{Conclusions and outlook}\label{conclusions}
The sustainable energy transition requires methods to bridge the gap between global plans and implications and processes on the local level. Techniques for downscaling of global decarbonization scenarios to finer scales will become increasingly important in the future. Thereby, energy-policy makers have to rely on the meaningfulness of the downscaled values, which requires tailor-made downscaling techniques for the different fields of energy systems. This work emphasizes a downscaling technique for the residential and commercial heating sector, taking into account the technology-specific requirements of heat network infrastructure accounting for the highly efficient use of energy carriers. In particular, the proposed downscaling techniques reveal the potentials of centralized heat supply in four different European decarbonization scenarios on a high spatial granularity.\newline

Results indicate that centralized heat systems undergo a fundamental shift going beyond the decarbonization of the supplying energy mix. In particular, the reduction of heat density of centralized heat networks compared to today's networks poses massive challenges to heat supply companies and fundamentally jeopardizes associated business models. At the same time, however, the heat network infrastructure may play a crucial role in expected energy systems since both use of local energy sources efficiently and on a large scale (e.g., geothermal sources, waste incineration, waste heat from industry, etc.) and unburden the electricity sector taking into account the aim of high electrification of different energy services. These trade-offs should be given greater consideration in the future and may have implications for the regulation and benchmarking of heat supply companies that provide centralized heat network infrastructure.\newline

Future work may address improvements of the proposed downscaling technique for the heating sector, taking into account a finer scale of spatial granularity, extensions of the introduced benchmarking of centralized heat networks using indicator values in the context of heat sources and local characteristics, higher resolution of the heat generation technologies/sources in terms of requirements for heat network infrastructure, and a detailed cost-benefit analysis of the centralized heat systems obtained by the downscaling (e.g., distribution line capacities, connection capacities to the public grid, etc.). 


