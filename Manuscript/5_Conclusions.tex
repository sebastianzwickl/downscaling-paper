\section{Conclusions and recommendations}\label{conclusions}
Sustainable energy transition requires methods to bridge the gap between global decarbonization pathways and the resulting necessary measures at a local level. This work emphasizes the development of different downscaling \replaced{techniques}{algorithms}, which we apply to the Austrian heating sector (residential and commercial) under several \replaced{scenarios}{storylines} in line with the Paris Agreement. We analy\replaced{z}{s}e results at the community \deleted{and grid }levels, considering technology-specific infrastructure requirements for the highly efficient usage of heat sources \added{in district heating networks}.\vspace{0.3cm}

We found that \replaced{the cost-effective and decarbonized heat supply in Austria in 2050 implies district heating in four different supply areas.}{the prioritized perspective of efficiency and local utilization of renewable heat sources implies substantial changes for the further development of district heating networks in the decarbonized Austrian heat supply toward 2050. This implies small-scale (1TWh) and large-scale (12TWh) district heating networks in terms of the amount of heat delivered.} The results demonstrate that \replaced{district heating continuous to be about cherry picking from}{particularly densely populated areas are still} beneficial supply areas \replaced{such as particularly densely populated areas albeit under different circumstances.}{for district heating networks and offer adequate heat densities. Nevertheless, most district heating networks in 2050 (seven of eight) will not reach the heat density benchmarks of today's networks and have a significant heat density gap.} \replaced{Because not all district heating networks in 2050 reach the heat density required for economic and technical efficiency from today’s techno-economic perspective and industry benchmarks. This heat density gap (mainly driven by a sigificant reduction of heat demands by building renovation) can be reduced by an increased allocation of large-scale heat pump genertion feeding into district heating.}{However, considering the increasing importance of local renewable heat sources feeding into district heating networks, we assume that these centralized networks will become required in the future and crucial in the decarbonization of the heating sector}.\vspace{0.3cm}

We anticipate our work as a starting point for discussing the role of \replaced{district heating}{centralized heat network infrastructure} for enabling large-scale, highly efficient and local integration of renewable heat sources such as \replaced{synthetic gas, hydrogen, geothermal sources and waste under lower total heat demands}{biomass/waste, hydrogen, ground-sourced heat pumps, or geothermal units}. \replaced{In addition, further research should follow on how local district heating networks and their heat densities or necessary shares of large-scale heat pump generation could be returned into the large-scale energy models in the sense of a feedback loop.}{In particular, we see a need for further research on the trade-off between local integration of heat sources and the cost-intensive deployment of district heating networks. Future work may elaborate on the increasing cooling demand and how the cooperative design of district heating and cooling networks can contribute to the profitability of centralized heating and cooling infrastructure.}