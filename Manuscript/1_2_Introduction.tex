\section{Introduction}
To implement the pathway in line with the Paris Climate Agreement \cite{agreement2015paris} as analy\replaced{z}{s}ed by the IPCC's \emph{Special Report on Global Warming of 1.5°C} (SR15) \cite{book}, the European Commission has set deep decarbonization targets together with national governments. In particular, the \replaced{\textit{EU Green Deal}}{"EU Green Deal"} describes the concrete goals in Europe, namely, a climate-neutral and resource-conserving economy and society \deleted{(see, e.g., }\cite{kemfert2019green}\deleted{)}. The overarching goal is to reduce carbon emissions to net-zero and hence achieve climate neutrality by 2050. The principles of a net-zero, decarbonized society are based on three key points: (i) reduction of the energy demand \deleted{(see, e.g., Oshiro et al. \cite{oshiro2021enabling} and Grubler et al. }\cite{grubler2018low}\deleted{)}, (ii) deployment and generation of renewable energy technologies \deleted{(see, e.g., Bakhtavar et al. }\cite{bakhtavar2020assessment}\deleted{)}, and (iii) an increase in efficiency regarding the provision of energy services and the associated optimal utilization of sustainable energy sources.\vspace{0.3cm}

To achieve these long-term ambitions, the European Commission recently presented \replaced{\textit{Fit for 55}}{"Fit for 55"}, a \deleted{concrete} roadmap \added{with specific actions and targets} \replaced{until}{to} 2030. This program commits to a \SI{55}{\%} reduction in CO\textsubscript{2} emissions in 2030 compared to \deleted{to} those in 1990 \cite{european_commission_european_2019}. The concrete measures affect almost all sectors of the energy system and should lead to a significant efficiency improvement and a massive overall reduction in fossil fuels. It implies, among others, binding annual targets to reduce energy consumption and to extend the already established EU emissions trading system (EU ETS) to new sectors. In addition to transportation, the building sector will be part of the EU ETS in the future. In the building sector, using the annual anchored emissions reduction, this means a defined roadmap to complete decarbonization of the heating and cooling demand. In this paper, we look at what deep decarbonization of building heating demand may look like in 2050 in Austria and the implications of the corresponding sustainable energy mix for \replaced{district heating}{centralized heating networks}.

\subsection{Implications of decarbonization on the heating sector}
The scope of changes required by 2030/2050 in the heating sector becomes even clearer at the national level. In Europe, the \deleted{average }share of renewable energies in the heating and cooling sector in 2018 is only just above \SI{20}{\%} on average\deleted{, for all EU member states} \cite{eurostat_reference}. \added{In Austria, it reaches 34\%}\deleted{It is, in fact, higher in some countries, for example, in Austria, it is above 34\%}. However, fossil fuels continue to dominate there as well. \added{In 2015, the heat demand for low-temperature heat services in Austria was approximately} \SI{96}{TWh}\added{ \cite{burandt2018genesys}. Thereby, natural gas, oil and coal account for almost 45\% of space heating and hot water demand in the residential building sector \cite{oesterreichsenergie}. The share of district heating reaches almost 15\% and more than \replaced{one million}{1,100,000} households are connected to district heating networks.}\footnote{\added{See }\ref{appendixC} \added{for a detailed overview of the Austrian heat market as well as references \cite{oesterreichsenergie} and \cite{buchele2015bewertung} for more details.}}.\deleted{To be even more specific for the heating sector, o}\added{Nevertheless, o}f the nearly 4,000,000 residential dwellings in Austria, more than \replaced{one million}{900,000} are heated with natural gas, and more than 500,000 with oil \cite{statistik_austria}. If these heating systems are converted to renewable energy supply by 2050, this corresponds to a retrofitting of 50,000 units per year, or more than 130 per day - only in Austria. To achieve this goal, measures that go beyond the electrification of heat supply are necessary, which may requre an expansion of district heating networks. This holds true even when substantial heat saving measures are installed such as better insulation of buildings \cite{jalil2018spatially}.\vspace{0.3cm}

\replaced{District heating is}{Centralized heating networks are} particularly advantageous for supplying densely populated or urban areas because of high heat densities \cite{inage2020development}. \added{Particularly in Europe, there are good conditions for district heating \cite{persson2019heat}.} In addition to heat density, the connection rate is a key factor determining the efficiency of district heating/cooling networks and thus their implementation. In Austria, a benchmark of \SI{10}{GWh \per km^2} at a connection rate of \SI{90}{\%} is currently used when deciding whether to supply an area with district heating\footnote{\url{http://www.austrian-heatmap.gv.at/ergebnisse/}}. This reference value is in line with findings regarding district heating networks also from the Scandinavian region (Denmark, Sweden, and Finland) \cite{zinko2008district}. These are rough estimates, but they do allow an initial assessment of the economic viability or feasibility of a district heating network. In a detailed consideration and evaluation of district heating networks, numerous factors play a decisive role. \added{For example, the design and topology of district heating networks has a significant impact on their cost-effectivness \cite{nussbaumer2016influence}.}\deleted{Nussbaumer and Thalmann \cite{nussbaumer2016influence} thoroughly elaborate on the network design and its impact on the profitability of centralized heat networks.} \added{In addition, the cost-optimized heat supply is also influenced by the location of heat generation units/sources within the networks \cite{laasasenaho2019gis}.} \deleted{In their study, Laasasenaho et al. \cite{laasasenaho2019gis} emphasize the optimal location of heat generation units/sources within centralized heat networks, enabling a cost-optimized heat supply.} \deleted{Gopalakrishnan and Kosanovic \cite{gopalakrishnan2014economic} focus on the optimal heat generation technology dispatch.} When examining the economic viability of district heating networks, building renovation measures must also be taken into account \deleted{(see, e.g., }\cite{andric2018impact}\deleted{ and \cite{rabani2021achieving})}. \replaced{A recent study shows}{Hietaharju et al. \cite{hietaharju2021stochastic} recently show in their analysis} that a $2-3$\SI{}{\%} building renovation rate per year results in a $19-28$\SI{}{\%} decrease of the long-term district heating demand \added{\cite{hietaharju2021stochastic} which consequently also reduces the heat densities of networks.}. \deleted{This also reduces the heat density.} However, studies show that a reduction in heat density is not necessarily a barrier to district heating networks \cite{persson2011heat}. \added{For example, energy taxes which can certainly be expected in the future (e.g., higher taxes on fossil fuels) can improve the profitability of sparse district heating networks \cite{reidhav2008profitability}.}\deleted{Reidhav and Werner \cite{reidhav2008profitability} show how energy taxes can improve the profitability of sparse district heating networks in Sweden.} Following these considerations and in light of ambitious CO\textsubscript{2} reduction targets, it can also be assumed that the rising CO\textsubscript{2} price can have an effect similar to the energy tax. \replaced{Naturally}{Of course}, this is valid only in the case of deep decarbonization of the generation mix feeding into \replaced{district}{centralized} heat\added{ing} networks. \added{In general, there are a variety of alternatives to decarbonize the energy mix of district heating networks. Among others, geothermal \cite{kyriakis2016towards}, biomass \cite{di2014low}, waste \cite{hiltunen2020highly} and heat recovery from industrial excess heat \cite{buhler2017industrial} are likely to be the primary heat sources in sustainable district heating networks.}\deleted{Di Lucia and Ericsson \cite{di2014low} show that biomass significantly contributed to the decarbonization of the district heating network and replaced fossil fuels in the feed-in generation mix in Sweden. In their multi-criteria study, Ghafghazi et al. \cite{ghafghazi2010multicriteria} also identify wood pellets as the optimal system option for fueling district heating networks.} Eventually, the increasing cooling demand and the co-design of \replaced{district}{centralized networks for} heating and cooling \added{networks} can also increase the economic viability of these and counteract the reduction of heat density from an economic point of view \cite{zhang2021economic}.

\subsection{Implications of large-scale numerical model results at the local level}
For quantifiying solutions of complex planning problems, researchers use numerical models. In general, these models strike a balance between complexity and aggregation. Integrated assessment models (IAMs) are large numerical models covering complex interrelationships between climate, society, economics, policy, and technology \cite{dowlatabadi1995integrated}. \added{Particularly, IAM contribute to the understanding of global energy decarbonization pathways \cite{wilkerson2015comparison}}.\deleted{Wilkerson et al. \cite{wilkerson2015comparison} and van Vuuren et al. \cite{van2016carbon} deal with IAMs and their role in understanding global energy decarbonization pathways.} \replaced{Evaluating and discussing IAM involves}{Schwanitz \cite{schwanitz2013evaluating} evaluates IAMs of global climate change and discusses}, among others, the appropriate level of regional (spatial) aggregation of countries in the modeling analysis \added{\cite{schwanitz2013evaluating}}. Generalizing this aspect reveals an aspect already known but essential in the context of large numerical models. It becomes necessary for modelers to set priorities regarding the level of detail, which inevitably creates trade-offs in the analysis regarding the granularity of temporal, spatial, and other dimensions \cite{gargiulo2013long}. \added{Accordingly, }\deleted{Gambhir et al. \cite{gambhir2019review} also highlight this aspect of aggregation bias in their critical review of IAMs. They propose, among others, that }IAMs should \deleted{be} increasingly be supplemented with other models and analytical approaches \added{\cite{gambhir2019review}}. Not least for this reason, large-scale detailed energy systems models also play a significant role in the analysis of energy systems in the context of climate change. Compared to IAMs, they more strongly emphasize the level of detail in terms of techno-economic characteristics. However, the lack of granularity remains: these global systems models consider only a highly aggregated spatial resolution. To name just two selected approaches, \deleted{Capros et al. \cite{capros2012model} (}PRIMES\deleted{)} \added{\cite{capros2012model}} and \deleted{Löffler et al. \cite{loffler2017designing} (}GENeSYS-MOD\deleted{)} \added{\cite{loffler2017designing}} \replaced{are}{provide} energy system models focusing on the European energy system with a spatial resolution at the country level. Further approaches are needed to disaggregate results obtained at the country level to finer scales, such as districts, neighborhoods, and other local levels. In this context, \deleted{Backe et al. \cite{backe2021heat} provided }a novel approach in the context of merging local activities/behavior in sustainable local communities into a large energy system model (bottom-up linkage) \added{is presented in \cite{backe2021heat}}. In \replaced{this}{their} study, \deleted{they integrated }local flexibility options \added{are integrated} into the global energy system model EMPIRE, which provides, in principle, only country-level resolution. This and other work confirms the emerging trend of making top-down and bottom-up linkages between different spatial-temporal levels of resolution to drive decarbonization across all sectors.\vspace{0.3cm}

\subsection{Objective and contribution of this work}
Against this background, the core objective of this work is downscaling European decarbonization scenarios of the heating sector to the community \replaced{or}{/} distribution grid level serving end-users in 2050. In particular, downscaling considers the highly efficient and local use of sustainable heat sources in \replaced{district}{centralized} heat\added{ing} networks (e.g., \added{geothermal sources, }co-firing \added{synthetic gas and }hydrogen in cogeneration plants and large-scale waste utilization\deleted{, etc.}). In addition, the topography of district heating networks is of particular importance and plays a crucial role in applied downscaling. This allows estimates of realistic \deleted{and cost-effective} decarbonized district heating networks in 2050 to be obtained, which can be compared with existing networks. Thereby, the heat density of district heating networks serves as a comparative indicator and permits a rough estimation of the changes needed for \replaced{district}{centralized} heating networks considering the 1.5°C climate target. An Austrian case study is conducted, downscaling the \added{cost-effective} results of the heating sector in 2050 from the large numerical energy system model GENeSYS-MOD, from the country to the community\replaced{or}{/} distribution grid levels.\vspace{0.3cm}

The method applied \added{(section \ref{methodology})} consists of three different scenario-independent downscaling techniques. In the first technique, proportional downscaling uses population as a stylized proxy (section \ref{pop}). In the second, a sequential downscaling approach is presented, disaggregating from the country level to the sub-region level. Thereby, the population density and infrastructure requirements of heat \added{sources/generation }technologies serve as additional criteria in the downscaling (section \ref{alg1}). Finally, an iterative downscaling algorithm is presented. The algorithm applies benchmarking based on graph-theory. It computes \replaced{district heating}{centralized heat supply} at the local (community) level, see section \ref{alg2}. Section \ref{results} presents and discusses the results of this work. Section\deleted{s \ref{res:1} and} \ref{res:2} show\added{s} heat generation by source at different spatial levels. Sections \ref{res:3} and \ref{res:4} present \replaced{district heating networks}{centralized heat networks} at a high spatial granularity. Section \ref{res:5} synthesizes the results of \replaced{district}{centralized} heat\added{ing} networks and compares heat densities of \replaced{district heating}{centralized heat networks} in 2050 with today's values. \added{Section \ref{sens:hp} presents a sensitivity analysis of the heat density of district heating regarding the allocation of heat generation by heat pumps feeding into district heating networks. }Section \ref{conclusions} concludes this work and provides an outlook for future work.