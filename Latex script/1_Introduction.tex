\section{Introduction}

\begin{itemize}
	\item \textcolor{magenta}{Which downscaling method for the energy technology generation comes closer to predicting the nodal heat demand in the residential and commercial sector accounting for energy technologies' infrastructure needs on a high spatial granularity?}
	
	\item \textcolor{magenta}{What, if any, is the impact of using such downscaling methods to dissaggregating the nodal heat demand on benchmarking network-based heat service provision focusing in particular on the district heating network?}
	
\end{itemize}
have displayed good simulation performance based on the large scale features and patterns of synoptic activities, but they are unable to capture the meso-scale and micro-scale physical processes that influence local climate variables.
the simulation suffer substantial from bias in specific and detailed regions
to bridge this spatial gap 

terms of statistical downscaling methods, the empirical relation
ships
between large-scale low-resolution climate variables (pre
dictors)
and local high-resolution parameters (predictands) in a
particular domain are established using statistical approaches, and
then applied to GCM outputs [41]. Due to the advantages of easy
implementation, fast computing speed, and low time consumption,
statistical downscaling methods are widely used in climate change
projections, in which large numbers of GCMs with various sce
narios
are often considered [42,43].